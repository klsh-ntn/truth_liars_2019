\documentclass[12pt]{article}

\usepackage{hyperref} % гиперссылки

\usepackage{tikz} % картинки в tikz
\usetikzlibrary{arrows.meta} % tikz-прибамбас для рисовки стрелочек подлиннее

\usepackage{microtype} % свешивание пунктуации

\usepackage{array} % для столбцов фиксированной ширины

\usepackage{indentfirst} % отступ в первом параграфе

\usepackage{sectsty} % для центрирования названий частей
\allsectionsfont{\centering}

\usepackage{amsmath} % куча стандартных математических плюшек
\usepackage{amssymb} % символы
\usepackage{amsthm} % теоремки

\usepackage{comment} % добавление длинных комментариев

\usepackage[top=2cm, left=1.2cm, right=1.2cm, bottom=2cm]{geometry} % размер текста на странице

\usepackage{lastpage} % чтобы узнать номер последней страницы

\usepackage{enumitem} % дополнительные плюшки для списков
%  например \begin{enumerate}[resume] позволяет продолжить нумерацию в новом списке

\usepackage{caption} % что-то делает с подписями рисунков :)

\usepackage{qcircuit} % для рисовки квантовых диаграмм
\usepackage{physics} % бракеты

\usepackage{answers} % разделение условий и ответов в упражнениях


\usepackage{fancyhdr} % весёлые колонтитулы
\pagestyle{fancy}
\lhead{Правдюки и Лжецы}
\chead{}
\rhead{КЛШ-2019}
\lfoot{}
\cfoot{}
\rfoot{\thepage/\pageref{LastPage}}
\renewcommand{\headrulewidth}{0.4pt}
\renewcommand{\footrulewidth}{0.4pt}



\usepackage{todonotes} % для вставки в документ заметок о том, что осталось сделать
% \todo{Здесь надо коэффициенты исправить}
% \missingfigure{Здесь будет Последний день Помпеи}
% \listoftodos — печатает все поставленные \todo'шки



\usepackage{booktabs} % красивые таблицы
% заповеди из докупентации:
% 1. Не используйте вертикальные линни
% 2. Не используйте двойные линии
% 3. Единицы измерения - в шапку таблицы
% 4. Не сокращайте .1 вместо 0.1
% 5. Повторяющееся значение повторяйте, а не говорите "то же"



\usepackage{fontspec} % что-то про шрифты?
\usepackage{polyglossia} % русификация xelatex

\setmainlanguage{russian}
\setotherlanguages{english}

% download "Linux Libertine" fonts:
% http://www.linuxlibertine.org/index.php?id=91&L=1
\setmainfont{Linux Libertine O} % or Helvetica, Arial, Cambria
% why do we need \newfontfamily:
% http://tex.stackexchange.com/questions/91507/
\newfontfamily{\cyrillicfonttt}{Linux Libertine O}

\AddEnumerateCounter{\asbuk}{\russian@alph}{щ} % для списков с русскими буквами
\setlist[enumerate, 2]{label=\asbuk*),ref=\asbuk*}

%% эконометрические сокращения
\DeclareMathOperator{\Cov}{Cov}
\DeclareMathOperator{\Arg}{Arg}
\DeclareMathOperator{\Corr}{Corr}
\DeclareMathOperator{\Var}{Var}
\DeclareMathOperator{\E}{\mathbb{E}}
\def \hb{\hat{\beta}}
\def \hs{\hat{\sigma}}
\def \htheta{\hat{\theta}}
\def \s{\sigma}
\def \hy{\hat{y}}
\def \hY{\hat{Y}}
\def \v1{\vec{1}}
\def \e{\varepsilon}
\def \he{\hat{\e}}
\def \z{z}
\def \hVar{\widehat{\Var}}
\def \hCorr{\widehat{\Corr}}
\def \hCov{\widehat{\Cov}}
\def \cN{\mathcal{N}}
\let\P\relax
\DeclareMathOperator{\P}{\mathbb{P}}



\usepackage[bibencoding = auto,
backend = biber,
sorting = none,
style=alphabetic]{biblatex}

\addbibresource{truth_liars.bib}



% делаем короче интервал в списках
\setlength{\itemsep}{0pt}
\setlength{\parskip}{0pt}
\setlength{\parsep}{0pt}




\Newassociation{sol}{solution}{solution_file}
% sol --- имя окружения внутри задач
% solution --- имя окружения внутри solution_file
% solution_file --- имя файла в который будет идти запись решений
% можно изменить далее по ходу
\Opensolutionfile{solution_file}[all_solutions]
% в квадратных скобках фактическое имя файла

% магия для автоматических гиперссылок задача-решение
\newlist{myenum}{enumerate}{3}
% \newcounter{problem}[chapter] % нумерация задач внутри глав
\newcounter{problem}[section]

\newenvironment{problem}%
{%
\refstepcounter{problem}%
%  hyperlink to solution
     \hypertarget{problem:{\thesection.\theproblem}}{} % нумерация внутри глав
     % \hypertarget{problem:{\theproblem}}{}
     \Writetofile{solution_file}{\protect\hypertarget{soln:\thesection.\theproblem}{}}
     %\Writetofile{solution_file}{\protect\hypertarget{soln:\theproblem}{}}
     \begin{myenum}[label=\bfseries\protect\hyperlink{soln:\thesection.\theproblem}{\thesection.\theproblem},ref=\thesection.\theproblem]
     % \begin{myenum}[label=\bfseries\protect\hyperlink{soln:\theproblem}{\theproblem},ref=\theproblem]
     \item%
    }%
    {%
    \end{myenum}}
% для гиперссылок обратно надо переопределять окружение
% это происходит непосредственно перед подключением файла с решениями



\theoremstyle{definition}
\newtheorem{definition}{Определение}



\begin{document}

\tableofcontents{}

\section*{Цель}




\setcounter{section}{0}
\section{Правдюки и Лжецы. Знакомство}

Лжец всегда лжёт. Правдюк всегда говорит правду. Игрок равновероятно говорит правду или лжёт. 

\begin{enumerate}
  \item На острове живут правдюки и лжецы. 
  Ты встретил аборигена и спросил его, кем он является. Что ответит абориген? 
  
  \item Ты вышел на дорогу, соединяющую город Лжевск и город Правдюковск. И разговорился с встреченным абориген. 
  Как за один вопрос узнать, в какой стороне находятся какой город?
  
  \item На острове живут правдюки, лжецы и игроки. Два аборигена, Андрей и Борис, сказали следующие утверждения:
  Aндрей: Борис — правдюк.
  Борис: Андрей — не правдюк.

  Кем могли быть Андрей и Борис? Сколько утверждений могут быть истинными?


  \item Андрей и Борис живут на острове лжецов и правдюков. Андрей говорит «Мы оба — одного вида». 
  Борис говорит «Мы аборигены разных видов».
  
  Что можно сказать про личности Андрея и Бориса?

  \item Андрей и Борис живут на острове лжецов и правдюков. Андрей говорит «Мы оба — лжецы». 
  
  Что можно сказать про личности Андрея и Бориса?


  \item На острове живут правдюки и лжецы, всего 10 человек. Они сели в круг. 
  Двое из них заявили: «Оба моих соседа – лжецы», а остальные восемь заявили: «Оба моих соседа – правдюки». 
  
  Сколько правдюков могло быть среди этих 10 человек? 


  \item Дама с собачкой пришла в магазин и спросила, сколько стоит диван. 
  В магазине было двое продавцов — Андрей и Борис. Андрей ответил, что диван стоит 600 монет.
  Затем Борис сказал, что Андрей любую цифру завышает в три раза.
  Затем Андрей сказал, что Борис занижает любую цифру в 12 раз.

  Сколько нужно Даме с собачкой заплатить за диван?

  \item На острове живут правдюки и лжецы. Все аборигены разного роста и разного возраста. Каждый абориген сделал два заявления:
  «На острове нет и десяти человек выше меня». «По крайней мере сто человек старше меня».

  Сколько аборигенов живёт на острове?


  \item На острове живут 100 правдюков и 100 лжецов, у каждого из них есть хотя бы один друг. 
  Однажды утром каждый житель произнес либо фразу «Все мои друзья – правдюки», либо фразу «Все мои друзья – лжецы», 
  причем каждую из фраз произнесло ровно 100 человек. 
  
  Найдите наименьшее возможное число пар друзей, один из которых правдюк, а другой – лжец. 



\end{enumerate}


\setcounter{section}{1}
\section{Правдюки и Лжецы. Продолжение}

Лжец всегда лжёт. Правдюк всегда говорит правду. Игрок равновероятно говорит правду или лжёт. 

\begin{enumerate}
  \item На острове живут лжецы и правдюки, всего 2001 человек. 
  Каждый житель острова заявил: «Среди оставшихся жителей острова более половины — лжецы». 
  Сколько лжецов на острове? 

  \item Все жители острова либо правдюки, либо лжецы. Путешественник встретил пятерых аборигенов. 
  На его вопрос: «Сколько среди вас правдюков?» первый ответил: «Ни одного!», а двое других ответили: «Один». 
  Что ответили остальные?

  \item Встретились несколько аборигенов (каждый — либо лжец, либо — правдюк), и каждый заявил всем остальным: «Вы все — лжецы». 
  Сколько правдюков было среди них?

  \item На острове живут лжецы и правдюки. 
  Каждый из них сделал по два заявления: «Среди моих друзей – нечётное количество рыцарей» и «Среди моих друзей – чётное количество лжецов». 
  Чётно или нечётно количество жителей острова? 

  \item Состоялся матч по футболу 10 на 10 игроков между командой лжецов и командой правдюков. 
  После матча каждого игрока спросили: «Сколько голов ты забил?» 
  Некоторые участники матча ответили «один», Миша сказал «два», некоторые ответили «три», а остальные сказали «пять». 
  
  Лжёт ли Миша, если правдолюбы победили со счётом  20 : 17? 
\end{enumerate}



\setcounter{section}{2}
\section{Всеобщность знания}
\begin{enumerate}
\item На острове живут пять Учеников и Гуру\footnote{Сама Гуру — голубоглазая блондинка.}. У двух Учеников зелёные глаза, а у трёх Учеников — карие.
Общаться Ученики не могут, увидеть цвет своих глаз — тоже не могут. Однако каждое утро 
они собираются вместе, и каждый видит всех остальных. 

Покинуть остров можно только на корабле, который отходит каждый вечер. 
Каждый Ученик, выяснивший цвет своих глаз, покидает остров ближайшим рейсом.

Однажды утром Гуру собрала всех Учеников и объявила им «Среди вас есть хотя бы один зеленоглазый». 

\begin{enumerate}
  \item Кто из Учеников покинет остров и на какой день?
  \item Как изменится ответ, если Гуру объявит «Среди вас есть хотя бы один кареглазый»?
  \item Как изменится ответ, если Гуру объявит «Среди вас есть хотя бы один кареглазый и хотя бы один зеленоглазый»?
  \item Какую новую информацию сообщила Гуру Ученикам?
  Казалось бы, ничего нового?
\end{enumerate}


\item На острове живут Ученики и Гуру. 
Общаться Ученики не могут, увидеть цвет своих глаз — тоже не могут. Однако каждое утро 
они собираются вместе, и каждый видит всех остальных. 

Покинуть остров можно только на корабле, который отходит каждый вечер. 
Каждый Ученик, выяснивший цвет своих глаз, покидает остров ближайшим рейсом.

Однажды утром Гуру собрала всех Учеников и сделала им объявление. 

Определи в каждом случае, кто из Учеников покинет остров и когда. 

\begin{enumerate}
  \item «Среди вас есть хотя бы один зеленоглазый», 50 зеленоглазых и 50 кареглазых.
  \item «Зеленоглазых больше, чем кареглазых, а кареглазых больше двух», 100 зеленоглазых и 50 кареглазых.
  \item «Зеленоглазых не меньше 10», 50 зеленоглазых и 50 кареглазых.
  \item «Зеленоглазых не меньше 10 и не больше 75», 50 зеленоглазых и 50 кареглазых.
  \item «Количество зеленоглазых не делится на 17», 100 зеленоглазых.
  \item «Есть хотя бы один зеленоглазый, хотя бы один кареглазый», 50 зеленоглазых, 50 кареглазых и 1 красноглазый.
\end{enumerate}


\end{enumerate}



\setcounter{section}{3}
\section{Передача информации}

\begin{enumerate}
\item У Холмса и Ватсона есть 10 предполагаемых дат покушений на королеву: 2 января, 5 января, 
3 февраля, 4 февраля, 6 февраля, 1 марта, 2 марта, 4 марта, 1 апреля, 3 апреля.
Информатор сообщил информацию Холмсу и Ватсону по частям: Холмсу — месяц покушения, а Ватсону — день.

Затем между Холмсом и Ватсоном состоялся следующий диалог:

Холмс: Мне неизвестна дата покушения, но я знаю, что и ты не знаешь.

Ватсон: Теперь я знаю дату.

Холмс: Теперь я тоже знаю.

Когда планируется покушение на королеву?


\item Встречаются два приятеля:

Андрей: Ну как дела, как живешь?

Борис: Все хорошо, растут два сына дошкольника.

Андрей: Сколько им полных лет?

Борис: Произведение их возрастов равно количеству голубей возле этой скамейки.

Андрей: Этой информации мне недостаточно.

Борис: Старший похож на мать.

Андрей: Теперь я знаю ответ на твой вопрос. 

Сколько полных лет сыновьям?


\item Гуру загадала два натуральных числа, возможно одинаковых, от 1 до 11. 
Гуру сообщила Ученику Андрею сумму, а Ученику Борису — произведение чисел. 
Между Учениками состоялся следующий диалог:

Андрей: Я не знаю этих чисел.

Борис: Я это знал.

Андрей: Тогда я знаю эти числа.

Борис: Тогда и я знаю! 

Какие это могли быть числа?


\item Гуру загадала два последовательных натуральных числа. 
Гуру сообщила Ученику Андрею одно число, а Ученику Борису — второе число. 
Каждый Ученик знает, что числа соседние. Между Учениками состоялся следующий диалог:

Андрей: Я не знаю твоего числа.

Борис: Я тоже не знаю твоего числа.

Андрей: Теперь я знаю.

Какие это могли быть числа?

\item Гуру загадала два последовательных натуральных числа не больше 10. 
Гуру сообщила Ученику Андрею одно число, а Ученику Борису — второе число. 
Каждый Ученик знает, что числа соседние. Между Учениками состоялся следующий диалог:

Андрей: Я не знаю твоего числа.

Борис: Я тоже не знаю твоего числа.

Андрей: Теперь я знаю.

Какие это могли быть числа?



\begin{comment}
\item Альберт и Бернард только что познакомились с Шерил. Они хотят знать, когда у неё день рождения. 
Шерил предложила им десять возможных дат: 15 мая, 16 мая, 19 мая, 17 июня, 18 июня, 14 июля, 16 июля, 14 августа, 15 августа и 17 августа. 
Затем Шерил сказала Альберту месяц своего рождения, а Бернарду — день. После этого состоялся диалог.

Альберт: Я не знаю, когда у Шерил день рождения, но я знаю, что Бернард тоже не знает.

Бернард: Поначалу я не знал, когда у Шерил день рождения, но знаю теперь.

Альберт: Теперь я тоже знаю, когда у Шерил день рождения. 

Когда у Шерил день рождения?
\end{comment}


\item Гуру загадала два натуральных числа, возможно одинаковых. 
Гуру сообщила Ученику Андрею сумму, а Ученику Борису — произведение чисел. 
Между Учениками состоялся следующий диалог:

Андрей: Я не знаю этих чисел.

Борис: Я это знал. Сумма чисел меньше 14. 

Андрей: Я это знал. Однако теперь я знаю загаданные числа!

Борис: Тогда и я знаю!

Какие это могли быть числа?


\item Гуру загадала два натуральных числа, возможно одинаковых. 
Гуру сообщила Ученику Андрею сумму, а Ученику Борису — произведение чисел. 
Между Учениками состоялся следующий диалог:

Андрей: Ты не можешь посчитать сумму.

Борис: Да, не мог. Зато после твоих слов могу! Сумма равна 136. 

Какие это могли быть числа?

\begin{comment}
\item Два детектива расследуют Загадочное убийство. 

Возможны девять вариантов:

\begin{tabular}{cccccccccc}
\toprule
Вариант & А & Б & В & Г & Д & Е & Ё & Ж & З \\
\midrule
Место & Москва & Москва & Москва & Москва & Москва & Москва & Москва & Москва & Москва & \\
Погода & Москва & Москва & Москва & Москва & Москва & Москва & Москва & Москва & Москва & \\
Орудие & Москва & Москва & Москва & Москва & Москва & Москва & Москва & Москва & Москва & \\
\bottomrule
\end{tabular}

Черновик задачки по мотивам Common knowledge, Geanakoplos. Статья из handbook of game theory.

\end{comment}

\item На острове живут три зеленоглазых Ученика и Гуру\footnote{Сама Гуру — голубоглазая блондинка.}.
Ученики не общаются между собой и ни один из них не знает свой цвет глаз.
Однажды Гуру собрала Учеников вместе и объявила им: «Хотя бы один из вас — зеленоглазый».
Затем она спрашивает по очереди каждого из учеников (Андрея, Бориса, Вову, потом снова Андрея и так далее): 
«Знаешь ли ты свой цвет глаз?».

Что будут отвечать Ученики?


\item У султана было два мудреца: Али и Вали. Однажды султан сказал мудрецам: «
Я задумал два числа, возможно одинаковых. 
Оба они целые, каждое больше единицы, но меньше ста. 
Я перемножил эти числа и результат сообщу Али и при этом Вали я скажу сумму этих чисел. 
Если вы и вправду так мудры, как о вас говорят, то сможете узнать исходные числа».
Султан сказал Али произведение, а Вали – сумму. Мудрецы задумались. Первым нарушил молчание Али.

Али: Я не знаю этих чисел.

Вали: Я это знал.

Али: Тогда я знаю эти числа.

Вали: Тогда и я знаю! 

И мудрецы сообщили пораженному султану задуманные им числа. Назовите эти числа.

\end{enumerate}

\section{Шах и Ахмет}

Ефим Чеповецкий

Однажды старый грозный шах

Вошёл в дворцовый зал

И, грусти не тая в глазах,

Придворным так сказал:

- Мы, мудрый шах Махмуд-палван,

О том скорбим душой,

Что пухнет наша голова

От мудрости большой.

Кто нынче шуткой облегчит

Страданья мудреца,

Тому дадим в награду чин

Придворного глупца.



- О, шах, я пред тобой пигмей, -

Сказал слуга Ахмед. –

Сравниться с мудростью твоей

Не может белый свет,

И всё ж прогнать осмелюсь я

Твою печаль долой,

Лишь отгадай, что у меня

В кармане под полой?

И если сможешь отгадать,

Что груши там лежат,

То половину их отдать

Тебе я буду рад.

На рынке не найти таких,

Прекрасен плод любой,

А отгадаешь сколько их,

То весь десяток твой… -

И мудрый шах склонил чело,

И долго ждал народ,

Пока сказал он: - Их число

Мой казначей сочтёт.

Но ты скажи мне сам, Ахмед,

Слуга покорный мой,

Что за таинственный предмет

Ты прячешь под полой?


\section{Загоночная контрольная}

\begin{enumerate}
  \item Все жители острова либо правдюки, либо лжецы. Путешественник встретил пятерых аборигенов. 
  На его вопрос: «Сколько среди вас правдюков?» первый ответил: «Ни одного!», а двое других ответили: «Один». 
  Что ответили остальные?

  \item Альберт и Бернард только что познакомились с Шерил. Они хотят знать, когда у неё день рождения. 
  Шерил предложила им десять возможных дат: 15 мая, 16 мая, 19 мая, 17 июня, 18 июня, 14 июля, 16 июля, 14 августа, 15 августа и 17 августа. 
  Затем Шерил сказала Альберту месяц своего рождения, а Бернарду — день. После этого состоялся диалог.
  
  Альберт: Я не знаю, когда у Шерил день рождения, но я знаю, что Бернард тоже не знает.
  
  Бернард: Поначалу я не знал, когда у Шерил день рождения, но знаю теперь.
  
  Альберт: Теперь я тоже знаю, когда у Шерил день рождения. 
  
  Когда у Шерил день рождения?
  

  \item На острове живут три Ученика: зеленоглазые Андрей и Борис и кареглазый Владимир. А ещё на острове живёт Гуру.
  Ученики не общаются между собой и ни один из них не знает свой цвет глаз.
  Однажды Гуру собрала Учеников вместе и объявила им: «Хотя бы один из вас — зеленоглазый».
  Затем она спрашивает по очереди каждого из учеников (Андрея, Бориса, Вову, потом снова Андрея и так далее): 
  «Знаешь ли ты свой цвет глаз?».
  
  Что будут отвечать Ученики?


\end{enumerate}



\section{Лог. КЛШ-2019}

\begin{enumerate}
  \item Было 15 школьников. Обратил внимание на идеи: перебор вариантов для 
  видов аборигенов, оформленный в табличку. Что можно сделать после решения задачи?
  Подумать, есть ли другой способ решения? Более быстрый? Более универсальный?
  Поменять задачу! Что будет, если заменить в задаче А на Б? 
  Не бояться сделать первый шаг. Попробовать конкретный вопрос на аборигенах,
  пусть даже он будет неоптимальный. Попробовать ввести дополнительные облегчающие предположения в задачу.
  На примере задачи два: предположим, что в Лжевске живут одни лжецы, а в Правдюковске — только те, кто говорят правду. 
  Предположим, что мы видили откуда пришёл абориген. Задача решается и без этих предположений :)
  Хорошая мысль — похвалить себя после решения задачи. С обсуждениями решили 1 - 4. Напоминал школьникам, что всегда можно решать вперёд. 
  \item Одна школьница выбрала другой курс, вместо неё пришла другая. Задачка про цену дивана и двух продавцов. Небольшим перебором. Идея: предложи пример! Пусть даже он не подходит
  в условие задачи! Решили системой уравнений. 
  Задача про людей разного возраста и роста. Предлагаем пример конфигурации. Затем идея экстремальных величин.
  Начали задачу про друзей. Сконструировали неоптимальный пример, подходящий в условие задачи. 
  \item Дорешали задачу про друзей. Начали всеобщность знания. Байка про ребёнка, маму и экспериментатора. Решили задачу 3.1. 
  \item Задача про всеобщность знания и остаток от деления на 17. Далее решили задачу про Холмса и двух приятелей.
  Обсудили, какие допущения мы делали в задаче про приятелей (целое число лет, разный возраст означает разное число лет).
  Обсудили, почему у французов натуральные числа начинаются с нуля, а в России — с единицы. Вероятно, из-за rez-de-chasse. 
  \item Задачи про передачу информации. Начали с идеи информационных множеств. Подбрасываем кубик. Что отличает рациональный игрок? 
  Что отличает житель племени пираху? Они считают один, два и много. Разобрали 4.4. про два соседних натуральных числа — соединяя линиями информационные множества.
  Почему-то не произносили слова «информационное множество». А просто говорили, что соединяем те состояния мира, которые игрок не может различить. 
  Затем разобрали задачу 4.8 про последовательные ответы учеников. Можно, наверное, было бы успеть рассказать динамику ответов с разных стартовых точек.
  Т.к. на самом деле схема игры следит сразу за всеми, но мы разобрали только с одного старта, когда все — зеленоглазые.
  Начали задачу 4.3. Поняли, что слипшиеся множества для Андрея — диагонали. А все допустимые — половинка квадрата плюс диагональ. 
  \item Дорешали задачу 4.3. У нас получилось два возможных ответа: (1,4) и (4,10). Несмотря на изначальный ужас школьников перед табличкой 11 на 11,
  после второй фразы всё вычеркивается довольно быстро. Далее играли в игру Шноля: ёжик (отгадывает), лев (всегда говорит правду),
  шакал (всегда лжёт), попугай (повторяет ответ предыдущего зверя), жираф (отвечает на свой предыдущий вопрос). Игры хватило на 35 минут после перерыва.
  Разделились на три команды по 5 человек. В каждой команде каждый по очереди был ёжиком. 
\end{enumerate}

В теховском файле \verb|\newpage| стоит, чтобы легко было скопировать секцию, для печати двух копий подряд на одном листе.
Это позволяет экономить бумагу и время при печати :)

\subsection{Плакат}





\Closesolutionfile{solution_file}

% для гиперссылок на условия
% http://tex.stackexchange.com/questions/45415
\renewenvironment{solution}[1]{%
         % add some glue
         \vskip .5cm plus 2cm minus 0.1cm%
         {\bfseries \hyperlink{problem:#1}{#1.}}%
}%
{%
}%



\section{Решения}
\input{all_solutions}


\section{Источники мудрости}

\todo[inline]{передалать потом в bib-файл}

\begin{enumerate}
\item \url{https://puzzling.stackexchange.com/}
\end{enumerate}

\printbibliography[heading=none]


\end{document}
